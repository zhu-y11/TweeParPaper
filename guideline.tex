\documentclass[11pt,a4paper]{article}
\usepackage[hyperref]{naaclhlt2018}
\usepackage{times}
\usepackage{tikz-dependency}
\usepackage{subcaption}
\aclfinalcopy
\newcommand{\yjcomment}[1]{\textcolor{orange}{[$_\mathrm{L}^\mathrm{Y}$#1]}}

\title{Annotation Guideline for \textsc{Tweebank v2}}
\author{Yijia Liu \\
	Harbin Institute of Technology \\
	{\tt yjliu@ir.hit.edu.cn} \\\And
	Yi Zhu \\
	University of Cambridge \\
	{\tt yz568@cam.ac.uk} \\\AND
}

\date{}

\begin{document}
\maketitle

\section{Introduction}
Annotating tweets is hard. 
Delivering consistent annotation is even harder.
The original annotation guideline in the universal dependencies (UD) project
gives good definition on the relations, but didn't cover some common
constructions in tweets.
What's more, ambiguities exist. 
Some constructions can either be explained by different guidelines in UD.

In this guideline, we list most common constructions
and annotation difficulties we encountered when creating \textsc{Tweebank v2}.
Our work includes:
\begin{itemize}
	\item specify what guideline to be applied when constructions that are
	uncommon in UD guideline.
	\item specify what guideline to be applied when there are ambiguities in
	applying the guidelines.
\end{itemize}

In this guideline, not only we touch the syntax part of the annotation,
but also the whole pipeline from sentence segmentation to POS tagging. 
\section{Non-syntactic Tokens}

In this section, we give the definition of the non-syntactic tokens
we mentioned in the paper.

\subsection{Retweet Construction}

\begin{figure}[t]
	\centering
	\small
	\begin{dependency}[edge slant=2, text only label, label style=above]
		\begin{deptext}
			RT \& @user \& : \& wow \\
			\texttt{X} \& \texttt{X} \& \texttt{PUNCT} \& \texttt{INTJ} \\
		\end{deptext}
		\deproot{4}{root}
		\depedge{4}{1}{discourse}
		\depedge{1}{2}{discourse}
		\depedge{1}{3}{punct}
	\end{dependency}
	\caption{An example for the conventional retweet construction.}\label{fig:rt}
\end{figure}

We consider the special construction in tweet: 
\[
\underline{\textit{RT @user :}}
\]
as the \textit{retweet construction}.
Figure \ref{fig:rt} illustrates the most standard retweet construction.
\yjcomment{copy our discussion from the paper here.}

Besides the discussion in the syntactic aspect, 
retweet construction marks the begining of a sentence and used to
determine the boundary of a sentence in tweets (see Section \ref{sec:sent-seg} for
more details).
However, there are still cases where retweet construction is unconventional, 
as shown in the following examples:
\[
\underline{\textit{RT @user1 : RT @user2 it s hilarious}}
\]
We consider the second as a retweet construction even without the comma.

\subsection{Referential URL}

\begin{figure}[t]
	\centering
	\small
	\begin{dependency}[edge slant=2, text only label, label style=above]
		\begin{deptext}
			it \& s \& hilarious \& URL \\
			PRON \& AUX \& ADJ \& X\\
		\end{deptext}
		\deproot{3}{root}
		\depedge{3}{1}{nsubj}
		\depedge{3}{2}{cop}
		\depedge{3}{4}{\textbf{list}}
	\end{dependency}
	\caption{An example for referential URL.}\label{fig:ref-url}
\end{figure}

Figure \ref{fig:ref-url} shows an example for the \textit{referential URL}.
We consider a URL as the referential URL when it meets the following conditions:

\begin{itemize}
	\item The URL is used as a supplementary resource for the tweet.
	\item When removed, the syntax of the rest part of the tweet doesn't change.
	\item The URL is not an appositional modifier of a noun phrase (NP). 
	Figure \ref{fig:non-ref-url}
	shows an example for a URL that was used as an appositional modifier.
\end{itemize}

For a referential URL, we analyze its syntactic head in the same way as
what UD does for \textit{punct}(uation) but use the \textit{list}
relation.

\paragraph{Cases of Non-referential URLs.}
\begin{figure}[t]
\centering
\small
\begin{subfigure}[t]{0.45\columnwidth}
	\centering
	\begin{dependency}[edge slant=2, text only label, label style=above]
	\begin{deptext}
		Photo \& : \& URL \\
		\texttt{NOUN} \& \texttt{PUNCT} \& \texttt{X} \\
	\end{deptext}
	\deproot{1}{root}
	\depedge{1}{2}{punct}
	\depedge{1}{3}{\textbf{appos}}
	\end{dependency}
\end{subfigure}
\begin{subfigure}[t]{0.45\columnwidth}
	\centering
	\begin{dependency}[edge slant=2, text only label, label style=above]
	\begin{deptext}
		click \& URL \\
		\texttt{VERB} \& \texttt{X} \\
	\end{deptext}
	\deproot{1}{root}
	\depedge{1}{2}{\textbf{obj}}
\end{dependency}
\end{subfigure}
\caption{Examples for non-referential URL.}\label{fig:non-ref-url}
\end{figure}

We consider the following cases as non-referential URLs:
\begin{itemize}
	\item \underline{\textit{watch : \textbf{URL}}} 
	In this case, we consider the URL as an object of \textit{watch}.
	\item \underline{\textit{click for details: \textbf{URL}}}
	In this case, we consider the URL as an appositional modifier of \textit{details}.
\end{itemize}

\paragraph{Cases of Referential URLs with Modifiers.}
\begin{figure}[t]
	\centering
	\small
	\begin{dependency}[edge slant=2, text only label, label style=above]
		\begin{deptext}
			video \& : \& URL \& via \& @user \\
			\texttt{NOUN} \& \texttt{PUNCT} \& \texttt{X} \& \texttt{ADP} \& \texttt{PROPN} \\
		\end{deptext}
		\deproot{1}{root}
		\depedge{1}{2}{punct}
		\depedge{1}{3}{list}
		\depedge{3}{5}{\textbf{nmod}}
		\depedge{5}{4}{case}
	\end{dependency}
	\caption{An example for URL that has modifier.}\label{fig:ref-w-mod}
\end{figure}
In most cases, referential URLs don't have modifiers.
The following cases are the only exceptions:
\begin{itemize}
	\item \underline{\textit{watch : \textbf{URL} via YouTube}}
	We consider \textit{via YouTube} as the nominal modifier of the URL
	(see Figure \ref{fig:ref-w-mod} for an example).
	\item \underline{\textit{click for details: \textbf{URL1} \textbf{URL2}}}
	We consider the second \textit{URL} as an conjunction of the first one.
\end{itemize}

\subsection{Topical Hashtag, Sentiment Emoticon, and Truncated Words}

\begin{figure}[t]
	\centering
	\small
	\begin{dependency}[edge slant=2, text only label, label style=above]
		\begin{deptext}
			I \& come \& conclusion \& that \& ... \& \#topsecret \\
			\texttt{PRON} \& \texttt{VERB} \& \texttt{NOUN} \& \texttt{SCONJ} \& \texttt{PUNCT} \& \texttt{X}\\
		\end{deptext}
		\deproot{2}{root}
		\depedge{2}{1}{nsubj}
		\depedge{2}{3}{obj}
		\depedge{3}{4}{ccomp}
		\depedge[edge unit distance=2ex]{2}{5}{punct}
		\depedge[edge unit distance=2ex]{2}{6}{\textbf{discourse}}
	\end{dependency}
	\caption{An example for topical hashtag.}\label{fig:hashtag}
\end{figure}

\textit{Topical hashtag}, \textit{sentiment emoticon}, and \textit{truncated words}
are defined in the same way as referential URL.
But we use \textit{discourse} to analyze its syntactic head.
Figure \ref{fig:hashtag} shows an example for topical hashtag.

Truncated URL.

\section{Multiple Sentences in One Tweet}\label{sec:sent-seg}
When encountering a raw tweet, 
the first thing we would do is to segment the tweet to sentences.
However, due to the highly varied styles of tweet, 
it might not be obvious to distinguish which segment is a sentence.
Here we propose several cases:
\begin{enumerate}
	\item Standard punctuations function the same in the normal text,
	i.e. period, exclamation, question mark are the mark of the end of the sentence,
	whereas comma, colon, ellipse mark do not split the tweet;
	\item If there is a retweet construction in the middle of the tweet,
	the RT mark is considered to segment the sentence and is the beginning of a new sentence;
	\item If we can distinguish clearly sentences from tweets, 
	but they are not split by the punctuations or RT. 
	we then treat them as a single sentence with \textit{parataxis};
\end{enumerate}

\begin{figure}[t]
	\centering
	\small
	\begin{dependency}[edge slant=2, text only label, label style=above]
		\begin{deptext}
			Yall \& sholl \& is \& quiet \& !! \& SPEAK \& UP \& lol \& RT \& user \& : \& wow \\
%			\texttt{PRON} \& \texttt{VERB} \& \texttt{NOUN} \& \texttt{SCONJ} \& \texttt{PUNCT} \& \texttt{X}\\
		\end{deptext}
		\deproot{4}{\textbf{root}}
		\depedge{4}{1}{nsubj}
		\depedge{4}{2}{aux}
		\depedge{4}{3}{cop}
		\depedge{4}{5}{punct}
		\deproot{6}{\textbf{root}}
		\depedge{6}{7}{compound:prt}
		\depedge{6}{8}{discourse}
		\deproot{12}{\textbf{root}}
		\depedge{12}{9}{discourse}
		\depedge{9}{10}{discourse}
		\depedge{9}{11}{punct}
	\end{dependency}
	\caption{An example for a tweet with multiple sentences. 
		Retweet  construction is also treated as a sentence delimiter.}\label{fig:multi-seg}
\end{figure}

\begin{figure}[t]
	\centering
	\small
	\begin{dependency}[edge slant=2, text only label, label style=above]
		\begin{deptext}
			i \& ca \& n't \& login \& no \& i \& do \& n't \& know \& y \& nvr \& c \& nus \& offer \\
			%			\texttt{PRON} \& \texttt{VERB} \& \texttt{NOUN} \& \texttt{SCONJ} \& \texttt{PUNCT} \& \texttt{X}\\
		\end{deptext}
		\deproot{4}{\textbf{root}}
		\depedge{4}{1}{nsubj}
		\depedge{4}{2}{aux}
		\depedge{4}{3}{part}
		\depedge{4}{9}{parataxis}
		\depedge{9}{5}{discourse}
		\depedge{9}{6}{nsubj}
		\depedge{9}{7}{aux}
		\depedge{9}{8}{part}
		\depedge{9}{10}{ccomp}
		\depedge[edge unit distance=2.5ex]{4}{12}{parataxis}
		\depedge{12}{14}{obj}
		\depedge{12}{11}{advmod}
		\depedge{14}{13}{compond}
	\end{dependency}
	\caption{An example for a tweet with multiple sentences. 
		Retweet  construction is also treated as a sentence delimiter.}\label{fig:single-seg}
\end{figure}

For the sentence in Figure \ref{fig:multi-seg},  there are 3 sentences.
RT splits the sentences and is the beginning of a new sentence.
While, for the sentence in Figure \ref{fig:single-seg},
there are only 1 sentence, although many sentences can be distinguished easily. 

\begin{figure}[t]
	\centering
	\small
	\begin{dependency}[edge slant=2, text only label, label style=above]
		\begin{deptext}
			i \& ca \& n't \& login \& no \& i \& do \& n't \& know \& y \& nvr \& c \& nus \& offer  \& :-/ \\
			%			\texttt{PRON} \& \texttt{VERB} \& \texttt{NOUN} \& \texttt{SCONJ} \& \texttt{PUNCT} \& \texttt{X}\\
		\end{deptext}
		\deproot{4}{root}
		\depedge[edge unit distance=1ex]{4}{9}{parataxis}
		\depedge[edge unit distance=1ex]{4}{12}{parataxis}
		\depedge[edge unit distance=1ex]{4}{15}{\textbf{discourse}}
	\end{dependency}
	\caption{An example for referential URL on parataxis sentences.}\label{fig:ref-url-on-para-sent}
\end{figure}

\subsection{Referential URL in Parataxis Sentences}
For those cases, we analyze the syntactic head of the referential URL
as the root word among the parataxis sentences.
See Figure \ref{fig:ref-url-on-para-sent} for an example.
This rule also applies to topical hashtags, sentiment emoticons, and truncated words.

Finally, we need to note that we didn't include sentence segmentation
as a step in the whole parsing pipeline.
The guideline only suggests how to 

\section{Named Entities}
%see newsgroup-groups.google.com\_emails\_ec0a1064de05e74b\_ENG\_20040929\_023800-0047
Compared with those in formal text, Twitter presents 
a wider range of entities.
We categories them as following.

\paragraph{Person name.} 
\begin{figure}[t]
	\begin{subfigure}[t]{0.3\columnwidth}
		\centering
		\small
		\begin{dependency}[edge slant=2, text only label, label style=above]
			\begin{deptext}
				Kevin \& Durant \\
				\texttt{PROPN} \& \texttt{PROPN} \\
			\end{deptext}
			\deproot{1}{}
			\depedge{1}{2}{flat}
		\end{dependency}
	\caption{}\label{fig:per-name:vanilla}
	\end{subfigure}
	\begin{subfigure}[t]{0.3\columnwidth}
		\centering
		\small
		\begin{dependency}[edge slant=2, text only label, label style=above]
			\begin{deptext}
				President \& Obama \\
				\texttt{PROPN} \& \texttt{PROPN} \\
			\end{deptext}
			\deproot{1}{}
			\depedge{1}{2}{flat}
		\end{dependency}
	\caption{}\label{fig:per-name:title}
	\end{subfigure}
	\begin{subfigure}[t]{0.3\columnwidth}
		\centering
		\small
		\begin{dependency}[edge slant=2, text only label, label style=above]
			\begin{deptext}
				Al \& - \& Zaman \\
				\texttt{PROPN} \& \texttt{punct} \& \texttt{PROPN} \\
			\end{deptext}
			\deproot{1}{}
			\depedge{1}{2}{punct}
			\depedge{1}{3}{flat}
		\end{dependency}
		\caption{}\label{fig:per-name:punct}
	\end{subfigure}
\end{figure}
Person name is easy to identify.
For tokens in a person name, their POS tags are \texttt{PROPN} and
we treat the left-most token as the syntactic head
and link the rest of the tokens in the person name to this token (see Figure \ref{fig:per-name:vanilla}).
Here, we show some special cases in person names.
\begin{itemize}
	\item Person name with title, like \underline{\textit{President Obama}} (Figure \ref{fig:per-name:title});
	\item Person name with punctuation, like \underline{\textit{Al - Zaman}} (Figure \ref{fig:per-name:punct});
\end{itemize}

\paragraph{Group name.}
Mentions about groups (music group, sport group, and etc.) are common in tweets.
The names of music group can present complex syntactic structures,
e.g. \underline{\textit{One Direction}}, \underline{\textit{Zac Brown Band}},
and \underline{\textit{Rico Riders}}.
In this guideline, we propose to analyze the syntax using \textit{compound}.
However, if there is \texttt{ADP} in the group name like \textit{Eagles of Death Mental}, we analyze them as a normal noun phrase.

There are several group names presented in \textsc{Tweebank v2}. Here is a comprehensive list:
\textit{One Direction, Super Junior, 
	Higher Authority,
	Eagles of Death Mental, 
	*5 Second of Summer*,
	Carolina Panthers, Rico Riders, Manchester United, Real Madrid,
	49ers, Patriots, Heat, Celtics,
	Mariners,
	Nature Republic,
	The weekend,
	Kiss Daniel,
	Yung 4,
	Doncaster Phoenix,
	West Hartlepool
}.

\paragraph{Brand name.}
We treat brand name in the same way with group name.
Here is a comprehensive list of brand name in \textsc{Tweebank v2}:
\textit{Amazon, Apple, Louis Vuitton, Facebook, Twitter}.

\paragraph{Location name.}
\begin{figure}[t]
	\begin{subfigure}[t]{0.3\columnwidth}
		\centering
		\small
		\begin{dependency}[edge slant=2, text only label, label style=above]
			\begin{deptext}
				Santa \& Cruz \\
				\texttt{PROPN} \& \texttt{PROPN} \\
			\end{deptext}
			\deproot{2}{}
			\depedge{2}{1}{compound}
		\end{dependency}
		\caption{}\label{fig:loc-name:vanilla}
	\end{subfigure}
	\begin{subfigure}[t]{0.3\columnwidth}
		\centering
		\small
		\begin{dependency}[edge slant=2, text only label, label style=above]
			\begin{deptext}
				101 \& Drive \\
				\texttt{NUM} \& \texttt{PROPN} \\
			\end{deptext}
			\deproot{2}{}
			\depedge{2}{1}{nummod}
		\end{dependency}
		\caption{}\label{fig:loc-name:comp}
	\end{subfigure}
	\begin{subfigure}[t]{0.3\columnwidth}
		\centering
		\small
		\begin{dependency}[edge slant=2, text only label, label style=above]
			\begin{deptext}
				Cleveland \& , \& OH \\
				\texttt{PROPN} \& \texttt{punct} \& \texttt{PROPN} \\
			\end{deptext}
			\deproot{1}{}
			\depedge{1}{2}{punct}
			\depedge{1}{3}{appos}
		\end{dependency}
		\caption{}\label{fig:loc-name:punct}
	\end{subfigure}
\end{figure}
There are three major types of location name:
\begin{itemize}
	\item Vanilla location: like \textit{New York}, \textit{New York City}, and \textit{Santa Cruz}. 
	We use \textit{compound} to analyze their relations (Figure \ref{fig:loc-name:vanilla});
	\item Details location: like \textit{101 Drive} and \textit{Downtown Atlanta}.
	We use their original relation (Figure \ref{fig:loc-name:comp});
	\item Location name with punctuation: \textit{Cleveland , OH}, (Figure \ref{fig:loc-name:punct});
\end{itemize}

\paragraph{Product name.}
We analyze product name as a normal nominal phrase.
Thus, tokens like \textit{shoulder bag} are analyzed as \texttt{NOUN} rather than \texttt{PROPN}.
\begin{itemize}
	\item Apple Music, Didi Chuxing. Zapata Beer
	\item Asus Gaming PC
	\item Louis Vuitton Monogram Marelle Shoulder Bag
	\item Puzzle Sharks
	\item After School Snack
	\item CARVIN BX250 250 W Micro Bass Guitar Amp Amplifier Head / Compressor
	\item Lake Wind Advisory
	\item CARVIN BX250
	\item Australian Gelato
	\item Pokemon GO
	\item Windows live
\end{itemize}

Product name usually comes with suffix (version number).
If one token or span of tokens is considered as the suffix, 
it won't be analyzed as syntactic head, but a modifier of the noun phrase prefix.
If the suffix is a number (like \underline{\textit{Kindle \textbf{3}}}), we analyze the POS as \texttt{NUM} and \textit{nummod}.
If the suffix is a noun phrase (like \underline{\textit{Samsung Galaxy \textbf{S6}}}), we analyze the POS as \texttt{PROPN} and \textit{nmod:npmod}.
We need to note that a combination of letter and number as version number is treated as nominal.

\paragraph{Organization name.} 
We analyze organization name as a normal noun phrase.
We use \texttt{PROPN} as the POS tag.
We analyze most of the organization names as compound.
But for cases like \textit{Santander UK}.
\begin{itemize}
	\item Clinton Global Initiative University
	\item Santander UK
	\item Parag Milk Foods
	\item Alabama Banking Department
	\item Lady Tattoo Studio
	\item Hartlepool Mail
	\item Big East
\end{itemize}

\paragraph{Event name.}
\begin{itemize}
	\item Bush Tax Cuts
	\item Down to Lunch
	\item Stanley Cup
	\item The CJUSD District College and Career Fair
	\item \#AV/IT Leadership Summit
	\item FA Cup
\end{itemize}

\paragraph{Title name.}
We analyze the title name as a normal phrase.
\begin{itemize}
	\item Gone with the wind
	\item lone survivor
\end{itemize}

%We collected as list of group names presented in \textsc{Tweebank v2}.
%\begin{itemize}
%	\item \textit{product name}: Apple Music
%	\item \textit{event name}: normal
%	\item \textit{title name}: normal
%	\item \textit{brand name}: 
%	\item \textit{product name}: normal
%	\item \textit{suffix, version}: mixture of \textit{4s}. npmod, conjuction.
%	\item \textit{americans}:
%\end{itemize}
%
%\begin{table}[t]
%	\centering
%	\begin{tabular}{lr}
%		\hline
%		category & POS \\
%		\hline
%		The Eyenimal Cat Video Camera & \\
%		Monogram Marelle Shoulder Bag & \\
%		Apple Music & \\
%		The Power Ball & \\
%		Parag Milk Foods & \\
%		Darby Video App & \\
%		Moon Child glow kit & \\
%		Boise State & \\
%		Dickey Amendment & \\
%		lone survivor & \\
%		Clinton Global Initiative University & \\
%		Zapata Beer & \\
%		LADY PYRAMID & \\
%		Bush Tax Cuts & \\
% Asus Gaming PC
%		Friday Night Lights & \\
%		\hline
%	\end{tabular}
%\end{table}


\section{Special Constructions}

\subsection{List vs Parataxis vs Appos}

Tweets are full of commercials, music references (see Figure \ref{fig:singer-title}).
It's worth studying these constructions.
``Informal and web text often contains passages which are meant to be
interpreted as lists but are parsed as single sentences.'' \yjcomment{copied from UD}

\begin{table}[t]
	\centering
	\begin{tabular}{lrr}
		\hline
		pattern & relation & example \\
		\hline
		singer - title & list & Figure \ref{fig:singer-title} \\
		title - singer & parataxis & Figure \ref{fig:title-singer} \\
		quote - author & parataxis & \\
		author - quote & parataxis & \\
		agency : title & parataxis & \\
		\hline
	\end{tabular}
\caption{text}\label{tbl:list-construct}
\end{table}

In this guideline, we category these special cases in Table \ref{tbl:list-construct}.

\begin{figure}[t]
	\centering
	\small
	\begin{dependency}[edge slant=2, text only label, label style=above]
		\begin{deptext}
			Lady  \& Gaga \& - \& Bad \& Romance \\
			\texttt{PRORN} \& \texttt{PROPN} \& \texttt{PUNCT} \& \texttt{ADJ} \& \texttt{NOUN} \\
		\end{deptext}
		\deproot{1}{root}
		\depedge{1}{2}{flat}
		\depedge{1}{3}{punct}
		\depedge{5}{4}{amod}
		\depedge[edge unit distance=2.5ex]{1}{5}{\textbf{list}}
	\end{dependency}
	\caption{A music reference tweet.}\label{fig:singer-title}
\end{figure}

\begin{figure}[t]
	\centering
	\small
	\begin{dependency}[edge slant=2, text only label, label style=above]
		\begin{deptext}
			i \& 'd \& do \& anything \& - \& simple \& plan \\
			\texttt{PRON} \& \texttt{AUX} \& \texttt{VERB} \& \texttt{PRON} \& \texttt{PUNCT} \& \texttt{ADJ} \& \texttt{NOUN} \\
		\end{deptext}
		\deproot{3}{root}
		\depedge{3}{1}{nsubj}
		\depedge{3}{2}{aux}
		\depedge{3}{4}{obj}
		\depedge{3}{5}{punct}
		\depedge[edge unit distance=2.5ex]{3}{7}{parataxis}
		\depedge{7}{6}{amod}
	\end{dependency}
	\caption{An example for leading @username.}\label{fig:title-singer}
\end{figure}

%see weblog-blogspot.com\_rigorousintuition\_20060511134300\_ENG\_20060511\_134300-0004
%Refer newsgroup-groups.google.com\_emails\_ec0a1064de05e74b\_ENG\_20040929\_023800-0047
\paragraph{When to Use Appos?}
\begin{figure}[t]
	\centering
	\small
	\begin{dependency}[edge slant=2, text only label, label style=above]
		\begin{deptext}
			I \& liked \& a \& @YouTube \& video \& : \& URL \\
			\texttt{PRON} \& \texttt{VERB} \& \texttt{DET} \& \texttt{PROPN} \& \texttt{NOUN} \& \texttt{PUNCT} \& \texttt{X} \\
		\end{deptext}
		\deproot{2}{root}
		\depedge{2}{1}{nsubj}
		\depedge{2}{5}{ojb}
		\depedge{5}{3}{det}
		\depedge{5}{4}{nmod}
		\depedge{5}{6}{punct}
		\depedge{5}{7}{\textbf{appos}}
	\end{dependency}
	\caption{An example for leading @username.}\label{fig:case-02}
\end{figure}

In most the cases, we won't analyze the token with \textit{appos}.
See Figure \ref{fig:case-02}.

\subsection{Noun phrase or Copula Ellipse}
\begin{figure}[t]
	\centering
	\small
	\begin{dependency}[edge slant=2, text only label, label style=above]
		\begin{deptext}
			@username \& covered \& in \& magic \& ! \\
			PROPN \& VERB \& ADP \& NOUN \& PUNCT \\
		\end{deptext}
		\deproot{2}{root}
		\depedge{2}{1}{nsubj}
		\depedge{2}{4}{obl}
		\depedge{4}{3}{mark}
		\depedge{2}{5}{punct}
	\end{dependency}
	\caption{An example for leading @username.}\label{fig:case-05-1}
\end{figure}

\begin{itemize}
	\item feb\_jul\_16.1469260380
	\item feb\_jul\_16.1460628240
	\item feb\_jul\_16.1469497200
	\item feb\_jul\_16.1463541180
	\item feb\_jul\_16.1456268400
	\item feb\_jul\_16.1462312200
	\item feb\_jul\_16.1458837000
	\item feb\_jul\_16.1458423240
\end{itemize}

\subsection{Code Switching}

\subsection{Leading @user: vocative or nsubj}
\begin{figure}[t]
	\centering
	\small
	\begin{dependency}[edge slant=2, text only label, label style=above]
		\begin{deptext}
			@user1  \& @user2 \& Problem \& is \& how \& we \\
			\texttt{PRORN} \& \texttt{PROPN} \& \texttt{NOUN} \& \texttt{VERB} \& \texttt{ADV} \& \texttt{PRON} \\
		\end{deptext}
		\deproot{4}{root}
		\depedge{4}{1}{\textbf{vocative}}
		\depedge{4}{2}{\textbf{vocative}}
		\depedge{4}{3}{nsubj}
	\end{dependency}
	\caption{An example for leading @user.}\label{fig:vocative}
\end{figure}

\begin{figure}[t]
	\centering
	\small
	\begin{dependency}[edge slant=2, text only label, label style=above]
		\begin{deptext}
			@user  \& hiding \& from \& this \& tornado \& . \& How \& are \& you \& ? \\
			\texttt{PROPN} \& \texttt{VERB} \& \texttt{ADP} \& \texttt{DET} \& \texttt{NOUN} \& \& \texttt{ADV} \& \texttt{AUX} \& \texttt{PRON} \\
		\end{deptext}
		\deproot{2}{root}
		\depedge{2}{1}{\textbf{vocative}}
		\depedge{2}{5}{obl}
		\depedge{5}{3}{case}
		\depedge{5}{4}{det}
	\end{dependency}
	\caption{An example for ambiguous leading @user which can either be analyzed as \textbf{nsubj} or \textbf{vocative}.}\label{fig:abnormal-vocative}
\end{figure}

The way of recognizing if a @user token is \textit{vocative}.
See Figure \ref{fig:case-01}.


\section{Worth Noting}
\subsection{Have sth. to}
\begin{figure}[t]
	\centering
	\small
	\begin{dependency}[edge slant=2, text only label, label style=above]
		\begin{deptext}
			have \& time \& to \& review \& it \\
			VERB \& NOUN \& PART \& VERB \& PRON \\
		\end{deptext}
		\deproot{1}{root}
		\depedge{1}{2}{obj}
		\depedge{2}{4}{acl}
		\depedge{4}{3}{mark}
		\depedge{4}{5}{obj}
	\end{dependency}
	\caption{An example for leading @username.}\label{fig:case-04}
\end{figure}
See email-enronsent35\_01-0096.


\subsection{for real, fr.}
\begin{figure}[t]
	\centering
	\small
	\begin{dependency}[edge slant=2, text only label, label style=above]
		\begin{deptext}
			@username \& covered \& in \& magic \& ! \\
			PROPN \& VERB \& ADP \& NOUN \& PUNCT \\
		\end{deptext}
		\deproot{2}{root}
		\depedge{2}{1}{nsubj}
		\depedge{2}{4}{obl}
		\depedge{4}{3}{mark}
		\depedge{2}{5}{punct}
	\end{dependency}
	\caption{An example for leading @username.}\label{fig:case-06}
\end{figure}
See answers-20111108092643AAXe4lD\_ans-0039

\subsection{U.K.-loving}
\begin{figure}[t]
	\centering
	\small
	\begin{dependency}[edge slant=2, text only label, label style=above]
		\begin{deptext}
			U.K. \& - \& loving \& stocks \\
			PROPN \& PUNCT \& VERB \& NOUN \\
		\end{deptext}
		\deproot{4}{root}
		\depedge{4}{3}{?}
	\end{dependency}
	\caption{An example for leading @username.}\label{fig:case-08}
\end{figure}

\subsection{POS for lmao}
is \textsc{INTJ}

\subsection{POS for live}
\begin{itemize}
	\item feb\_jul\_16.1464410640
\end{itemize}

\subsection{Only one clause}
\begin{itemize}
	\item feb\_jul\_16.1468795800
\end{itemize}
Use the root in the clause.

\subsection{Additional adp}
\begin{itemize}
	\item feb\_jul\_16.1454771880
\end{itemize}

\subsection{, you know ,}
\begin{itemize}
	\item feb\_jul\_16.1463121540
\end{itemize}

\subsection{Title modifier}
Several references:
\begin{itemize}
	\item weblog-blogspot.com\_dakbangla\_20050311135387\_ENG\_20050311\_135387-0161
	\item email-enronsent16\_01-0055
\end{itemize}
\begin{itemize}
	\item feb\_jul\_16.1460481960
	\item feb\_jul\_16.1459860060
	\item feb\_jul\_16.1467549120
	\item feb\_jul\_16.1467056520
\end{itemize}

What about `One direction'?


\subsection{Special Token}
\begin{itemize}
	\item tbh: to be honest
	\item dm: direct message: can be noun or verb.
	\item tl: timeline: like dm, can be noun or verb.
\end{itemize}

\subsection{Date}
\begin{itemize}
	\item July 9th where 9th is a NOUN.
	\item April 2, 2016
\end{itemize}

\subsection{Title}
\begin{itemize}
	\item Arsenal chief XX
\end{itemize}

\subsection{title}
6 - game losing streak

pos should concern.
\begin{itemize}
	\item dm?
\end{itemize}

\subsection{Vocative or subject}

\end{document}